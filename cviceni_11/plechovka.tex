Víte, že plechovka má tvar válce a má mít objem jeden litr.
Chcete použít co nejméně plechu.
Jaký nejmenší povrch může taková plechovka mít?

\solution{
	\begin{itemize}

		\item  \textbf{Zavedeme značení:}
			\begin{itemize}
				\item  $v$ je výška
				\item  $r$ je poloměr podstavy
				\item  $P$ je plocha -- dvě podstavy a plášť: $P = 2\pi r^2 + 2\pi r v$
				\item  $V = 1$ je objem, který máme zadaný, ze střední víme, že $V = \pi r^2 v$.
			\end{itemize}

		\item  \textbf{Vyjádříme výšku jako funkci poloměru} (nebo naopak).
			$$v = \frac{1}{\pi r^2}$$
			kde $v, r \in [0, \infty)$.

		\item  \textbf{Minimalizujeme plochu jako funkci poloměru}
			$$P(r) = 2 \pi r^2 + 2 \pi r \frac{1}{\pi r^2}$$

			Derivujeme a najdeme nulové body derivace:
			$$P'(r) = 4 \pi r - \frac{2}{r^2} = 0$$
			$$r = \sqrt[3]{\frac{1}{2\pi}}$$

		\item  \textbf{Vyšetříme krajní body a nulové body derivace}
			Pokud se poloměr blíží nule nebo nekonečnu, pak se plocha také blíží nekonečnu.
			Minimum tedy nastává pro $v = 2r$ (takových plechovek moc není).

	\end{itemize}
}

