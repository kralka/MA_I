Najdete k následujícím funkcím jejich primitivní funkce (pro danou $f(x)$ najděte $F(x)$ takovou, že $F'(x) = f(x)$).

\begin{enumerate}

	\item  $f(x) = x^2 + 2x - 3$

		\solution{
			Uhodneme, že $F(x) = \frac{x^3}{3} + x^2 - 3x + c$ (kde $c \in \mathbb{R}$).
		}

	\item  $f(x) = e^x - e^{-x}$

		\solution{
			$F(x) = e^x + e^{-x} + c$
		}

	\item  $f(x) = \sin(x) + \cos(x)$

		\solution{
			$F(x) = -\cos(x) + \sin(x) + c$
		}

	\item  $f(x) = x e^{-x^2}$

		\solution{
			Vzpomeneme si na derivaci složené funkce $\left( f(g(x)) \right)' = f'(g(x)) g'(x)$.
			Pokud obě strany integrujeme, dostáváme:
			$$f(g(x)) = \int \left( f(g(x)) \right)' dx = \int f'(g(x)) g'(x) dx$$

			Chceme tedy chytře uhodnout vnitřní a vnější funkci.
			Nechť $f'(x) = e^x$ a $g(x) = -x^2$.
			Pak primitivní funkce k $f'(g(x))g'(x) = e^{-x^2} (-2x)$ je funkce $f(g(x)) = e^{-x^2}$.

			Snadno tedy dopočítáme, že
			$$\int x e^{-x^2} dx = -\frac{1}{2} e^{-x^2} + c$$
		}

	\item  $f(x) = x \sin(x)$

		\solution{
			Vzpomeneme si na pravidlo o derivaci součinu:
			$$(f(x)g(x))' = f(x) g'(x) + f'(x) g(x).$$
			Zintegrováním obou stran a převedením dostaneme:
			$$\int f(x) g'(x) dx = f(x) g(x) - \int f'(x) g(x) dx$$
			Tomuhle se říká integrace per-partes (po částech).

			$$\int x \sin(x) dx = x (-\cos(x)) - \int -\cos(x) dx = - x \cos(x) + \sin(x) + c$$
		}

	\item  $f(x) = x e^x$

		\solution{
			$$ \int x e^x dx = x e^x - \int e^x dx = x e^x - e^x + c = (x-1)e^x + c$$
		}

	\item  $f(x) = \ln(x)$

		\solution{
			$$ \int \ln(x) \cdot 1 dx = x \ln(x) - \int 1 dx = x \ln(x) - x + c$$
		}

\end{enumerate}

