Řešte následující rovnice a nerovnice v $\mathbb{R}$:

\begin{enumerate}

	\item  $\log(x^2 - 25) = \log(2x + 10)$

		\solution{
			Víme, že funkce logaritmus je definovaná jen pro kladná reálná čísla.
			Dostáváme tedy pro každou stranu rovnice jednu podmínku:
			\begin{itemize}
				\item Pro levou stranu rovnice:
					\begin{align*}
						x^2 - 25 &> 0 \\
						x^2 &> 25 \\
						x &\in (-\infty, -5) \cup (5, \infty)
					\end{align*}
				\item Pro pravou stranu rovnice:
					\begin{align*}
						2x + 10 &> 0 \\
						x &> 5 \\
						x &\in (5, \infty)
					\end{align*}
			\end{itemize}
			Obě strany rovnice jsou tedy definované jen pro interval $(5, \infty)$, tedy všechna řešení která nalezneme tam musí patřit.

			Dále si uvědomíme, že logaritmus je prostá funkce a tedy můžeme psát:
			\begin{align*}
				\log(x^2 - 25) &= \log(2x + 10) \\
				x^2 - 25 &= 2x + 10 \\
				x^2 -2x - 35 &= 0 \\
				(x + 5)(x - 7) &= 0
			\end{align*}
			Tato kvadratická rovnice má dvě řešení: $x_1 = -5, x_2 = 7$, jen $x_2 = 7 \in (5, \infty)$ (tedy nás ani nenapadne dosadit $-5$ do jednoho z logaritmů a přemýšlet čemu se rovná logaritmus nuly -- v nule není logaritmus definován).

			Tedy naše původní rovnice má jediné řešení $x = 7$.
		}

	\item  $\log(x^2 + 1) = 2 \log(3 - x)$

		\solution{
			Napřed zase určíme, kde jsou obě strany definované:
			\begin{itemize}
				\item  Levá strana:
					\begin{align*}
						x^2 + 1 &> 0 \\
						x \in \mathbb{R}
					\end{align*}
				\item  Pravá strana:
					\begin{align*}
						3 - x > 0 \\
						x &\in (-\infty, 3)
					\end{align*}
			\end{itemize}
			Tedy hledáme řešení v množině $(-\infty, 3)$ (jen tam jsou všechny výrazy definované).

			Použijeme vlastnosti logaritmu k postupným úpravám:
			\begin{align*}
				\log(x^2 + 1) &= 2 \log(3 - x) \\
				\log(x^2 + 1) - 2 \log(3 - x) &= 0 \\
				\log(x^2 + 1) - \log\left((3 - x)^2\right) &= 0 \tag{$a \log x = \log (x^a)$} \\
				\log\left(\frac{x^2 + 1}{(3 - x)^2}\right) &= 0 \tag{rozdíl logaritmů je logaritmus podílu} \\
				\frac{x^2 + 1}{(3 - x)^2} &= 1  \tag{$\log x = 0$ jen pro $x = 1$} \\
				x^2 + 1 &= (3 - x)^2  \tag{$x \in (-\infty, 3)$, nedělíme ani nenásobíme nulou} \\
				x &= \frac{4}{3} \in (-\infty, 3)
			\end{align*}
			Jediné řešení tedy je $\frac{4}{3}$.
		}

	\item  $\frac{x-2}{2x-8} \geq 1$

		\solution{
			Podíl není definovaný pro $x = 4$.

			Jedničku převedeme na levou stranu rovnosti a po převodu na společný jmenovatel dostaneme:
			\begin{align*}
				\frac{x-2}{2x-8} &\geq 1 \\
				\frac{x-2 + 2x-8}{2x-8} &\geq 0 \\
				\frac{x-2 - 2x+8}{2x-8} &\geq 0 \\
				\frac{-x+6}{2x-8} &\geq 0 \\
			\end{align*}

			Poslední nerovnice jistě platí pokud jsou obě strany kladné \textbf{nebo} obě záporné:
			\begin{itemize}

				\item  $-x + 6 \geq 0$ a zároveň $2x - 8 > 0$ tedy $x \in (-\infty, 6] \cap [4, \infty) = (4, 6]$ (pozor na dělení nulou)

				\item  $-x + 6 \leq 0$ a zároveň $2x - 8 < 0$ tedy $x \in [6, \infty) \cap (-\infty, 4) = \emptyset$

			\end{itemize}
			Protože mezi podmínkami bylo nebo, řešení je spojení daných množin $x \in (4, 6] \cup \emptyset = (4, 6]$, což souhlasí s naší původní podmínkou $x \neq 4$.
		}

	\item  $||x - 2| + 1| \leq 5$

		\solution{
			Zaveďte substituci $y = |x - 2|$ a vyřešte $|y + 1| \leq 5$, pak odvoďte řešení pro původní rovnici~s~$x$.
		}

	\item  $\sin(2x) = \sin(x)$

	\item  $\sin(3x - 2) > \frac{1}{3}$

\end{enumerate}

