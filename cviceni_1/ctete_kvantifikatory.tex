Přečtěte následující a zjistěte, co znamenají:
\begin{enumerate}

	\item  Pro dané $x \in \mathbb{Z}$ definujeme
		$y \in \mathbb{Z}\colon
		(y \leq x \  \vee \  y \leq -x) \wedge
		\left( \forall z \in \mathbb{Z}\colon (z \leq x \vee z \leq -x) \Rightarrow z \leq y \right)
		$
		Co je $y$?

		\solution{ Absolutní hodnota $y = |x|$. }

	\item  Pro dané $a, b \in \mathbb{N}$ definujme $c \in \mathbb{N}$:
		\begin{align*}
			((\exists d \in \mathbb{N}\colon cd = a) \wedge (\exists d \in \mathbb{N}\colon cd = b)) \wedge
			\forall d \in \mathbb{N}\colon \left(
			((\exists e \in \mathbb{N}\colon de = a) \wedge (\exists e \in \mathbb{N}\colon de = b))
			\Rightarrow d \leq c \right)
		\end{align*}
		Co je $c$?

		\solution{
			Největší společný dělitel.
			Výrok $x$ je dělitelné $y$ zapíšeme jako: $\exists z\colon xz=z$.
		}

	\item  $A = \left\{ 1, 2, 3, \ldots, |A| \right\}$ Co je $A$?

		\solution{
			Nesmysl, taková množina není dobře definovaná.
			Respektive tuto definici splňuje libovolná množina prvních $k$ přirozených čísel.
		}

	\item  $\forall n \in \mathbb{N}\colon \exists! B \subseteq \mathbb{N}\colon n = \sum_{d \in B} 2^d$
		Co je $B$?

		\solution{
			Množina pozic s jedničkami v binárním zápisu čísla $n$.
		}

\end{enumerate}

