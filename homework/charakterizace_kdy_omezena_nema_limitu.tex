Nechť $(a_n)_{n=1}^{\infty}$ je omezená posloupnost.
Dokažte, že $(a_n)_{n=1}^{\infty}$ \emph{nemá} limitu právě tehdy, když existují aspoň dvě její podposloupnosti, které mají různou limitu.

\solution{
\begin{enumerate}

	\item  víme, že aspoň nějaká podposloupnost má limitu $a$.
		pokud celá $(a_n)$ nemá limitu $a$, pak $\exists \varepsilon > 0 \  \forall n_0 \in \mathbb{N} \  \exists n > n_0 \colon |a_n - a| \geq \varepsilon$.
		tedy existuje nekonečně mnoho prvků $(a_n)$, které jsou mimo $(a - \varepsilon, a + \varepsilon)$ a to je omezená podposloupnost, která má podposloupnost, která má limitu.

	\item  pokud existují dvě podposloupnosti, které mají různou limitu, pak $(a_n)$ nemá limitu (lehká modifikace https://iuuk.mff.cuni.cz/~tereza/teaching-files-19/analyza/lecture2.pdf věta 2.7).

\end{enumerate}

	Jde to dokázat pomocí věty z přednášky o hromadných bodech a limitě.
	Případně jste přišli i na různé kombinace předchozího a následujícího postupu nebo i na další jiné postupy, které sem psát nebudu.
	Zajímavějším typem důkazu byl ten následující:

	Podle Bolzano-Weierstrassovy věty existuje konvergentní podposloupnost.
	Značme nějakou takovou $(b_n)$, nechť $b = \lim b_n$.

	Z předpokladu $(a_n)$ nemá limitu, speciálně $b$ není její limitou. tedy:
	$$\exists \varepsilon > 0 \ \forall n_0 \in \mathbb{N} \  \exists n' \geq n_0\colon  | a_{n'} - b | \geq \varepsilon$$
	vytvoříme posloupnost $(c_n)$ následovně:
	$c_1 = a_1$
	nechť $c_n = a_m$ pro nějaké $m, n$ přirozené.
	použijeme výrok o neexistenci limity $(a_n)$, za $n_0$ dosadíme $m+1$, pak
	existuje $n' \geq m+1$, že $| a_{n'} - b | \geq \varepsilon$, položíme $c_{n+1} = n'$ (tedy vybíráme podposloupnost -- indexy které z $(a_n)$ vybereme rostou).

	$(c_n)$ je také omezená posloupnost, dle Bolzano-Weierstrassovy věty má konvergentní podposloupnost,
	$(d_n)$ a ta je i podposloupností $(a_n)$ a navíc má limitu různou od $b$ (není v intervalu $(b-\varepsilon, b+\varepsilon)$ pro naše pevné epsilon z předchozího odstavce).

	Tedy máme dvě podposloupnosti s různou limitou.

	Na co si dát pozor:
	\begin{itemize}

		\item  Potřebujete vybrat podposloupnost (tedy nekonečně mnoho prvků $(a)$ a tak aby jejich indexy byly rostoucí).

		\item  Pokud bychom neodrazili zbylé prvky od $b$, může nám vyjít stejná limita jako předtím.
			Uvažme posloupnost $1, 1/1, 0, 1, 1/2, 0, 1, 1/3, 0, 1, 1/4, 0, 1, 1/5, 0, \ldots$.

			\begin{itemize}

				\item  První aplikace Bolzano-Weierstrassovy věty nám může dát podposloupnost samých nul (limita 0).

				\item  Druhá aplikace Bolzano-Weierstrassovy věty na to, co zbyde, nám může dát podposloupnost
					$$1/2, 1/4, 1/8, 1/16, \ldots$$
					(limita 0).

				\item  Třetí aplikace Bolzano-Weierstrassovy věty na to, co zbyde, nám může dát podposloupnost
					$$1/3, 1/9, 1/27, 1/81, \ldots$$
					(limita 0).

				\item  A tak až do nekonečna.

			\end{itemize}

	\end{itemize}
}


