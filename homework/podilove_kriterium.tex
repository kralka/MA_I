% http://reseneulohy.cz/868/limita-posloupnosti---podilove-kriterium

Dokažte následující tvrzení:
Nechť $(a_n) \subset (0, \infty)$ je posloupnost kladných reálných čísel a nechť $N \in \mathbb{N}$ je takové, že $\exists q \in [0,1)$ že pro každé přirozené $n \geq N$ platí:
$$\frac{a_{n+1}}{a_n} \leq q < 1.$$

Dokažte, že:
\begin{enumerate}
	\item  Pro každé přirozené $n \geq N$ platí $a_n \leq a_N q^{n - N}$ (matematickou indukcí)
	\item  v důsledku čehož: $\underset{n\rightarrow\infty}{\lim} a_n = 0$.
\end{enumerate}

Nalezněte posloupnost \textbf{kladných} reálných čísel $(b_n)$ takovou, že:
\begin{itemize}
	\item  $b_n > 0$ pro všechna přirozená $n$
	\item  $\frac{b_{n+1}}{b_n} < 1$
	\item  $\underset{n\rightarrow\infty}{\lim} b_n > 0$
\end{itemize}

\solution{
	$$b_n = 1 - \frac{1}{n}$$

	Platí všechny tři body najednou.
	Ale tady není možné najít ono $q$ (tedy nemůžeme použít podílové kritérium a to je dobře, protože její limita není nulová).
	Limita $\lim_{n \rightarrow \infty} b_n = 1$.

	Proč jsme potřebovali to $q$?
	Protože díky němu jsme mohli říct, že další člen je nejvýš $q$-násobek předchozího, tedy to $q$ nám \uv{stlačuje} každý další člen blíž k nule.
}

