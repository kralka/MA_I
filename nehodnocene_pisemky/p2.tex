\documentclass[a4paper,10pt]{article}

\usepackage[utf8]{inputenc}
\usepackage{amsmath}
\usepackage{amsfonts}
\usepackage{amssymb}
\usepackage{amsthm}
% něco s indexem
% \usepackage{makeidx}
\usepackage[czech]{babel}
\usepackage{fontenc}
%\usepackage[dvips,pdftex,draft]{graphicx}
\usepackage{fancyhdr} %zahlavi a zapati
\usepackage{a4wide} % širší stránka
\usepackage{float} % obrázky na jedno místo
\usepackage{tikz}
\usepackage{tkz-euclide}
\usetikzlibrary{calc}
\usetikzlibrary{intersections}
\usetikzlibrary{arrows,automata}
\usepackage{tikz-3dplot}
\usepackage{hyperref}

%\usepackage[dvips,pdftex]{hyperref}
% solution macro, this does NOT prints the solution
\newcommand{\solution}[1]{}

\author{Karel Král}

\usepackage{graphicx}

\begin{document}
% Definitions
\newtheorem{theorem}{Věta}
\newtheorem{lemma}[theorem]{Lemma}
\newtheorem{conjecture}[theorem]{Domněnka}
\newtheorem{observation}[theorem]{Pozorování}
\newtheorem{corollary}[theorem]{Důsledek}
\newtheorem*{definition}{Definice}
%\theoremstyle{definition}\newtheorem*{define}{Definice}
%\vskip -90cm
\lhead{Nehodnocená písemka 2}
\chead{Jméno (nebo jen přezdívka):}
\pagestyle{fancy}
\pagenumbering{gobble} % bez čísel stránek


\paragraph{Příklad}[0 bodů]
Pokus se odpovědět na libovolnou podmnožinu z následujících otázek.
Vynasnažím se poskytnout zpětnou vazbu a přizpůsobit podle toho cvičení.
\begin{enumerate}
	\item  Co jsi nepochopil(-a) na přednášce / na cvičení?
	\item  Pokusíme se vymyslet důkaz Bolzano-Weierstrassovy věty (jiný, než někteří z vás viděli na přednášce).
\begin{theorem}[Bolzano-Weierstrassova]
	Nechť $(a_n)_{n = 1}^{\infty} \subset \mathbb{R}$ je omezená posloupnost, pak $(a_n)_{n = 1}^{\infty}$ má konvergentní podposloupnost.
	\label{thm:bolzanoWeierstrass}
\end{theorem}

Postupujte následovně a doplňujte co nejformálněji následující kroky:
\begin{enumerate}
	\item  Budete konstruovat posloupnost do sebe vnořených uzavřených intervalů. Začněte takovým, který obsahuje $a_n$ pro každé $n$.
	\item  Rozdělte svůj interval na dva poloviční délky. Rozmyslete, že aspoň jeden z nich obsahuje ``spoustu'' $a_n$. Pokračujte s tímhle.
	\item  Ukažte, limita délek našich intervalů jde k nule.
	\item  Použijte tvrzení, že průnik libovolného množství do sebe vnořených omezených uzavřených intervalů obsahuje aspoň jedno reálné číslo $x$.
	\item  Vyberte z posloupnosti $(a_n)_{n = 1}^{\infty}$ podposloupnost tak, že z každého zkonstruovaného intervalu vyberete jedno $a_n$ (dejte pozor, ať se vybraným prvkům zvětšují indexy).
	\item  Z definice dokažte, že $x$ z předminulého bodu je limita vybrané posloupnosti.
\end{enumerate}


\end{enumerate}

\end{document}
