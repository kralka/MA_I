\documentclass[a4paper,10pt]{article}

\usepackage[utf8]{inputenc}
\usepackage{amsmath}
\usepackage{amsfonts}
\usepackage{amssymb}
\usepackage{amsthm}
% něco s indexem
% \usepackage{makeidx}
\usepackage[czech]{babel}
\usepackage{fontenc}
%\usepackage[dvips,pdftex,draft]{graphicx}
\usepackage{fancyhdr} %zahlavi a zapati
\usepackage{a4wide} % širší stránka
\usepackage{float} % obrázky na jedno místo
\usepackage{tikz}
\usepackage{tkz-euclide}
\usetikzlibrary{calc}
\usetikzlibrary{intersections}
\usetikzlibrary{arrows,automata}
\usepackage{tikz-3dplot}
\usepackage{hyperref}

%\usepackage[dvips,pdftex]{hyperref}
% solution macro, this does NOT prints the solution
\newcommand{\solution}[1]{}

\author{Karel Král}

\usepackage{graphicx}

\begin{document}
%\vskip -90cm
\lhead{Nehodnocená písemka 1}
\chead{Jméno (nebo jen přezdívka):}
\pagestyle{fancy}
\pagenumbering{gobble} % bez čísel stránek


\paragraph{Příklad}[0 bodů]
Pokus se odpovědět na libovolnou podmnožinu z následujících otázek.
Vynasnažím se poskytnout zpětnou vazbu a přizpůsobit podle toho cvičení.
\begin{enumerate}
	\item  Proč studuješ informatiku MFF?
	\item  Co jsi nepochopil(-a) na přednášce / na cvičení?
	\item  Nechť $A$ je neprázdná podmnožina reálných čísel ($A \subseteq \mathbb{R}$, navíc $A \neq \emptyset$).
Víte, že neexistuje minimum $A$.
Zapište formálně pomocí matematických symbolů výrok \uv{$A$ má minimum} a pak ho znegujte.
Dokažte, že $A$ je nekonečně velká (aspoň spočetně velká).

\solution{
  \uv{$A$ má minimum}:
	$$\exists m \in A \  \forall a \in A \colon m \leq a$$

	Negace \uv{$A$ má minimum}:
	$$\forall m \in A \  \exists a \in A \colon m > a$$

	Víme, že $A$ je neprázdná, vezmeme tedy $a_0 \in A$ libovolné.
	\begin{itemize}
		\item  Z předpokladu víme, že $a_0$ není minimum, existuje tedy $a_1 \in A$ takové, že $a_1 < a_0$.
		\item  Z předpokladu víme, že $a_1$ není minimum, existuje tedy $a_2 \in A$ takové, že $a_2 < a_1$.
		\item  Z předpokladu víme, že $a_2$ není minimum, existuje tedy $a_3 \in A$ takové, že $a_3 < a_2$.
		\item  Z předpokladu víme, že $a_3$ není minimum, existuje tedy $a_4 \in A$ takové, že $a_4 < a_3$.
		\item  \ldots
	\end{itemize}
	Navíc podle tranzitivity platí $a_j < a_k$ pro každé dvě $j<k \in \mathbb{N}$.
	Takto jsme našli spočetně mnoho (pro každé přirozené číslo jedno) různých prvků $a_i$, které náleží do $A$.
	Množina $A$ má tedy nekonečně velkou podmnožinu a je tedy sama aspoň spočetně velká.

	Lze také dokazovat sporem.
	Nechť je konečná, pak má minimum (plyne z toho, že máme lineární uspořádání).
	Ale důkaz předchozí věty je stejně většinou stejný jako předchozí důkaz.
}


	\item  Určete definiční obor následující funkce $\frac{1}{\sqrt{5 - \log_2(x^2 - 9)}}$.

\solution{
	Logaritmus je definovaný jen pro kladná reálná čísla, tedy $x^2 - 9 > 0$, z čehož plyne první podmínka $x \in (-\infty, -3) \cup (3, \infty)$.

	Odmocnina je definovaná pro nezáporná čísla, ale dělíme jejím výsledkem, takže ani nesmí vyjít nula, dohromady tedy máme:
	\begin{align*}
		5 - \log_2(x^2 - 9) &> 0 \\
		5 &> \log_2(x^2 - 9) \tag{$2^y$ je rostoucí funkce} \\
		2^5 &> x^2 - 9 \\
		41 &> x^2
	\end{align*}
	Tedy druhá podmínka je: $x \in (-\sqrt{41}, \sqrt{41})$.

	Obě podmínky musí platit zároveň a dostáváme výsledek:
	$$x \in (-\sqrt{41}, -3) \cup (3, \sqrt{41}).$$

	Pozor, že ve všech řešeních netriviálně využíváte toho, že dvojkový logaritmus je rostoucí funkce nebo že dva na něco je rostoucí funkce.
}


\end{enumerate}

\end{document}
