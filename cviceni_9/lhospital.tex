Pomocí l'Hospitalova pravidla vypočítejte následující limity
\begin{enumerate}

	\item  $\underset{x \rightarrow 1}{\lim} \frac{x^2 + 2x - 3}{x^2 - 1}$

		\solution{
			Čitatel i jmenovatel konvergují k nule, použijeme l'Hospitalovo pravidlo (Věta~\ref{thm:lhospital}), konkrétně \uv{případ 0/0} (derivace jmenovatele je nenulová na malém prstencovém okolí jedničky):
			\begin{align*}
				\lim_{x \rightarrow 1} \frac{x^2 + 2x - 3}{x^2 - 1} &= \lim_{x \rightarrow 1} \frac{(x^2 + 2x - 3)'}{(x^2 - 1)'} \\
				&= \lim_{x \rightarrow 1} \frac{2x + 2}{2x} \tag{Věta o aritmetice limit~\ref{thm:aritmetika_limit_funkci}} \\
				&= 2
			\end{align*}
		}

	\item  $\underset{x \rightarrow \infty}{\lim} x \sin\left( \frac{1}{x} \right)$

		\solution{
			Napřed chytře upravíme (nekonečno krát nula se nám nelíbí, tak nulu napíšeme jako 1/nekonečnem), pak l'Hospitalíme (0/0, derivace jmenovatele je nenulová na nějakém malém okolí nekonečna):
			\begin{align*}
				\lim_{x \rightarrow \infty} x \sin\left( \frac{1}{x} \right) &= \lim_{x \rightarrow \infty} \frac{ \sin\left( \frac{1}{x} \right) }{1/x} \tag{l'Hospital 0/0} \\
				&= \lim_{x \rightarrow \infty} \frac{ \left(\sin\left( \frac{1}{x} \right)\right)' }{(1/x)'} \\
				&= \lim_{x \rightarrow \infty} \frac{ \cos\left( \frac{1}{x} \right) \frac{-1}{x^2} }{\frac{-1}{x^2}} \\
				&= \lim_{x \rightarrow \infty} \cos\left( \frac{1}{x} \right) \tag{Věta o aritmetice limit} \\
				&= 1
			\end{align*}
		}

	\item  $\underset{x \rightarrow 0}{\lim} \left( \frac{\sin(x)}{x} \right)^{\frac{1}{1-\cos(x)}}$

		\solution{
			Neumíme vyhodnotit, protože máme $\left( 0/0 \right)^\infty$.
			Nevíme co s tím, ale máme nějakou funkci na nějakou funkci, tak použijeme starý dobrý trik a všechno převedeme do exponentu a pak budeme moct l'Hospitalovat:
			\begin{align*}
				\lim_{x \rightarrow 0} \left( \frac{\sin(x)}{x} \right)^{\frac{1}{1-\cos(x)}} &= \lim_{x \rightarrow 0} e^{\frac{\ln\left( \frac{\sin(x)}{x} \right)}{1-\cos(x)}}
			\end{align*}

			Exponenciála je spojitá funkce, použijeme tedy větu o limitě složené funkce (Věta~\ref{thm:limita_slozene_fce}) a počítáme jen vnitřek, který rozložíme na něco, co už dopočítáme (a nejspíš jsme něco podobného viděli):
			\begin{align*}
				\lim_{x \rightarrow 0} \frac{\ln\left( \frac{\sin(x)}{x} \right)}{1-\cos(x)}
				&= \lim_{x \rightarrow 0}
					\left( \frac{\ln\left( \frac{\sin(x)}{x} \right)}{\frac{\sin(x)}{x} - 1} \frac{\frac{\sin(x)}{x} - 1}{x^2} \frac{x^2}{1-\cos(x)} \right) \tag{Věta o aritmetice limit} \\
				&=
					\left( \lim_{x \rightarrow 0} \frac{\ln\left( \frac{\sin(x)}{x} \right)}{\frac{\sin(x)}{x} - 1} \right)
					\left( \lim_{x \rightarrow 0} \frac{\frac{\sin(x)}{x} - 1}{x^2} \right)
					\left( \lim_{x \rightarrow 0} \frac{x^2}{1-\cos(x)} \right)
			\end{align*}
			\emph{
				Mohli jste místo tohohle rozpisu použít l'Hospitala (ověřte předpoklady)?
				Pokud jste mohli, dostane vás to někam (vyzkoušejte)?
			}

			Dopočítáme ty tři dílčí limity:
			\begin{itemize}

				\item  
					Použijeme větu o limitě složené funkce (Věta~\ref{thm:limita_slozene_fce}), kde použijeme, že na $P(0, 1/5)$ platí, že $\frac{sin(x)}{x} \neq 1$.
					Pak využijeme toho, že známe vnitřní limitu známe nebo dopočítáme jedním l'Hospitalem.
					Pak použijeme l'Hospitala na 0/0.
					\begin{align*}
						\lim_{x \rightarrow 0} \frac{\ln\left( \frac{\sin(x)}{x} \right)}{\frac{\sin(x)}{x} - 1} &= \lim_{x \rightarrow 1} \frac{\ln(x)}{x - 1} \\
						&= \lim_{x \rightarrow 1} \frac{1}{x} \\
						&= 1
					\end{align*}

				\item  Z definice funkce $\sin$ pomocí nekonečné řady je to celkem vidět.
					Obě části jdou k~nule, aplikujeme l'Hospitala na 0/0 (možná chcete zase l'Hospitalovat, abyste se přesvědčili, že čitatel limití k nule).
					\begin{align*}
						\lim_{x \rightarrow 0} \frac{\frac{\sin(x)}{x} - 1}{x^2} &= \lim_{x \rightarrow 0} \frac{\left( \frac{\sin(x)}{x} - 1 \right)'}{(x^2)'} \\
						&= \lim_{x \rightarrow 0} \frac{ \frac{x \cos(x) - \sin(x)}{x^2} }{2x} \tag{zjednodušíme, jinak budeme litovat} \\
						&= \lim_{x \rightarrow 0} \frac{ x \cos(x) - \sin(x) }{2x^3} \tag{l'H 0/0} \\
						&= \lim_{x \rightarrow 0} \frac{ -\sin(x) }{6x} \tag{l'H 0/0} \\
						&= -\frac{1}{6}
					\end{align*}

				\item  Známá limita (koukněte se na definici $\cos$) nebo můžeme zase dvakrát l'Hospitalovat.
					\begin{align*}
						\lim_{x \rightarrow 0} \frac{x^2}{1-\cos(x)} &= \lim_{x \rightarrow 0} \frac{2x}{\sin(x)} = \lim_{x \rightarrow 0} \frac{2}{\cos(x)} = 2
					\end{align*}

			\end{itemize}
			Dílčí limity vyšly a jejich součin dává smysl, mohli jsme tedy použít větu o aritmetice limit.
			\begin{align*}
				\lim_{x \rightarrow 0} \frac{\ln\left( \frac{\sin(x)}{x} \right)}{1-\cos(x)} &= - \frac{1}{3}
			\end{align*}

			Dosadíme do původní limity:
			\begin{align*}
				\lim_{x \rightarrow 0} \left( \frac{\sin(x)}{x} \right)^{\frac{1}{1-\cos(x)}} &= \lim_{x \rightarrow 0} e^{\frac{\ln\left( \frac{\sin(x)}{x} \right)}{1-\cos(x)}} = e^{-\frac{1}{3}}
			\end{align*}
		}

\end{enumerate}

