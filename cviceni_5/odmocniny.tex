Spočtěte limity:
\begin{itemize}

	\item  $\lim_{n \rightarrow \infty} \sqrt[n]{n}$ (spíš těžší teoretický bonus, můžete využívat dále),

		\solution{
			$\lim_{n \rightarrow \infty} \sqrt[n]{n} = 1$

			Platí $\forall n \in \mathbb{N}: \sqrt[n]{n}>1$, ale
			$\forall c>1 \  \exists n_0 \  \forall n \geq n_0\colon n < c^n$
			takže $\sqrt[n]{n}<c$.
			Takže nás může napadnout, že $1$ je ta správná limita.

			Uvažme posloupnost $a_n$ takovou, že $\sqrt[n]{n}=1+a_n$. Tato $a_n$ jsou kladná a my ukážeme, že jdou k $0$.

			Umocníme na $n$-tou a dostaneme
			$$n = (1+a_n)^n = \sum_{j=0}^n \binom{n}{j} a_n^j \geq \frac{n(n-1)}{2} a_n^2.$$

			Takže
			\begin{eqnarray*}
			n &\geq& \frac{n(n-1)}{2} a_n^2 \\
			1 &\geq& \frac{n-1}{2} a_n^2 \\
			\frac{2}{n-1} &\geq& a_n^2 \\
			\end{eqnarray*}
			a tím pádem máme $0 \leq a_n \leq \sqrt{\frac{2}{n-1}}$ a $\lim_{n \rightarrow \infty} a_n = 0$.
		}

	\item  $\lim_{n \rightarrow \infty} \sqrt[n]{2}$,

		\solution{
			Obdobně jako v předchozím příkladě usoudíme, že
			$\lim_{n \rightarrow \infty} \sqrt[n]{2} = 1$.
		}

	\item $\lim_{n \rightarrow \infty} \sqrt[n]{n^2 + 2^n}$,

		\solution{
			Pokusíme se odhadnout sdola konstantou a shora posloupností a použít větu o dvou policajtech (Věta~\ref{thm:dva_policajti}).

			\begin{itemize}

				\item  Nejprve $\sqrt[n]{n^2 + 2^n} \geq \sqrt[n]{2^n} = 2$ a máme dolní odhad.

				\item  Pro $n \geq 4$ je $2^n \geq n^2$ a tedy $\sqrt[n]{n^2 + 2^n}
					\leq \sqrt[n]{2 \cdot 2^n} = 2 \sqrt[n]{2}$ a jelikož posloupnost
					$\sqrt[n]{2}$ jde k $1$, máme horní odhad také $2$.

			\end{itemize}
		}

	\item $\lim_{n \rightarrow \infty} \sqrt[n]{a^n + b^n + c^n}$ pro $a,b,c$ nezáporná reálná.

		\solution{
			Předpokládejme $a\leq b \leq c$, pak podobně jako v předchozí úloze dostáváme
			$\sqrt[n]{a^n + b^n + c^n} \geq \sqrt[n]{c^n} = c$ jako dolní odhad a
			$\sqrt[n]{a^n + b^n + c^n} \leq \sqrt[n]{3c^n} \leq c\sqrt[n]{3}$ jako horní odhad jdoucí také k $c$.
		}

\end{itemize}

