Najděte primitivní funkci:

\begin{enumerate}

	\item $\int \frac{1}{1-x^2} \dx$

		\solution{
			Myšlenkou je rozložit podíl dvou polynomů na součet \uv{jednoduchých} zlomků.
			Za \uv{jednoduché} zlomky považujeme ty, které už umíme integrovat:
			\begin{itemize}

				\item  $$ \int \frac{1}{x} \dx = \ln(|x|)$$

				\item  Pro $k \in \mathbb{N}$, $k>1$:
					$$
						\int \frac{1}{x^k} \dx = -\frac{1}{(k-1)x^{k-1}} +C
					$$

				\item  Z přednášky:
					$$
						I_n(x) = \int \frac{1}{(x^2 + 1)^n} \dx
					$$
					pak
					\begin{align*}
						I_1(x) &= \arctan(x) \\
						I_{n+1} &= \frac{1}{2n(1+x^2)^n} + \left( 1 - \frac{1}{2n} \right) I_n
					\end{align*}

				\item  Poslední případ, který uvedeme:
					$$\int \left( \frac{1}{x^2 + \alpha x + \beta} \right)^j = \frac{I_j(y(x))}{\eta^{2j - 1}}$$
					kde $\eta = \sqrt{\gamma - \beta^2/4}$ a $y(x) = \frac{x}{\eta} + \frac{\beta}{2\eta}$.

				\item  Zbytek případů dostáváme pomocí více či méně jednoduchých substitucí.

			\end{itemize}

			Naštěstí se můžeme opřít o následující fakt (všechno podrobněji například ve skriptech doc. Klazara \url{https://kam.mff.cuni.cz/~klazar/maII.pdf}):
			každý polynom tvaru
			$$P(x) = x^k + \sum_{j = 0}^{k-1} p_j x^j$$
			kde $p_j \in \mathbb{R}$ se dá rozložit:
			$$P(x) = \prod_{j = 1}^{\ell} (x - \alpha_j)^{q_j} \prod_{j = 1}^{m} (x^2 + \beta_j x + \gamma_j)^{r_{j}}$$
			kde $q_j, r_j \in \mathbb{N}$.

			Pomocí tohoto tvrzení můžeme rozložit jmenovatel na předchozí tvar a pak vyjádřit zlomek jako součet jednoduchých zlomků.

			\textbf{Zpátky k našemu původnímu příkladu:}

			Nejprve rozložíme zlomek na jednodušší části, tzv. parciální zlomky
			$$\frac{1}{1-x^2} = \frac{1}{(1-x)(1+x)} = \frac{\alpha}{1-x} + \frac{\beta}{1+x}$$
			(všimněte si, že v čitateli máme jen konstanty, to je dáno tím, že čitatel a jmenovatel musí být nesoudělné).

			Kde $\alpha, \beta$ dopočteme zpětným převedením na společného jmenovatele a spočtením čitatele
			$$\frac{\alpha}{1-x} + \frac{\beta}{1+x} = \frac{\alpha(1+x)+\beta(1-x)}{(1-x)(1+x)} = \frac{(\alpha+\beta)+x(\alpha-\beta)}{1-x^2} = \frac{1}{1 - x^2}$$
			s toho dostáváme soustavu rovnic
			\begin{eqnarray*}
			\alpha+\beta &=& 1 \\
			\alpha-\beta &=& 0
			\end{eqnarray*}
			s řešením $\alpha=\beta=\frac{1}{2}$.
		
			Takže
			$$\int \frac{1}{1-x^2} \dx = \int \left( \frac{1}{2} \cdot \frac{-1}{x-1} + \frac{1}{2} \cdot \frac{1}{1+x} \right) \dx =
			\frac{1}{2} (\ln|x+1| - \ln|x-1|) +C$$
		}

	\item $\int \frac{x-2}{(x-1)^2} \dx$

		\solution{
			Opět rozložíme na parciální zlomky
			$$ \frac{x-2}{(x-1)^2} = \frac{\alpha}{x-1} + \frac{\beta}{(x-1)^2}$$
			$$ = \frac{\alpha(x-1)+\beta}{(x-1)^2} = \frac{(\beta-\alpha) + \alpha x}{(x-1)^2}$$
			a porovnáním s původním čitatelem dostáváme $\alpha=1$, $\beta=-1$
			$$\int  \frac{x-2}{(x-1)^2} = \int \left(\frac{1}{x-1} - \frac{1}{(x-1)^2} \right) \dx =
			\ln|x-1| - \frac{-1}{x-1} +C$$
		}

	\item $\int \frac{3}{x^2+2x+4} \dx$

		\solution{
			Tady rozklad nelze provést, neboť jmenovatel je ireducibilní polynom nad $\mathbb{R}$ (nemá reálné kořeny).
			Vyzkoušíme si tedy jednodušší verzi posledního \uv{kuchařkového} zlomku.

			Situace je podobná jako u funkce $\frac{1}{1+x^2}$, tak to zkusíme využít.

			Nejprve upravíme jmenovatele
			$$x^2+2x+4 = x^2+2x+1+3=(x+1)^2+3 = 3 \left(\left(\frac{x+1}{\sqrt{3}}\right)^2 +1\right)$$
			takže
			$$\int \frac{3}{x^2+2x+4} \dx = \int \frac{3}{3((\frac{x+1}{\sqrt{3}})^2 +1)} \dx$$
			a substituce
			\begin{align*}
				t &= \frac{x+1}{\sqrt{3}} \\
				\dt &= \frac{1}{\sqrt{3}} \dx
			\end{align*}
			dává
			$$ \int \frac{3}{3((\frac{x+1}{\sqrt{3}})^2 +1)} \dx = \int \frac{1}{t^2+1} \sqrt{3} \ dt = \sqrt{3} \arctan t = \sqrt{3} \arctan \frac{x+1}{\sqrt{3}} +C$$
		}

\end{enumerate}

