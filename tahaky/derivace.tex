\begin{theorem}
	O derivacích víte následující:
	\begin{enumerate}
		\item  $c' = 0$ (derivace konstanty je nula) \label{poucka:derivace_konstanty}
		\item  $(x^k)' = k x^{k-1}$ pro libovolné $k \in \mathbb{R}$ kdykoliv je toto definováno (bacha na dělení nulou při derivaci $(x^{1/2})' = \frac{1}{\sqrt{x}}$) \label{poucka:derivace_monomu}
		\item  $\sin'(x) = \cos(x)$ \label{poucka:derivace_sin}
		\item  $\cos'(x) = -\sin(x)$ \label{poucka:derivace_cos}
		\item  $\ln'(x) = 1/x$ pro $x>0$ \label{poucka:derivace_ln}
		\item  $\left( \frac{f(x)}{g(x)} \right)' = \frac{f'(x)g(x) - f(x)g'(x)}{g(x)^2}$ kdekoliv $g(x) \neq 0$
		\item  $(e^x)' = e^x$ \label{poucka:derivace_exponencialy}
		\item  Derivace je lineární operátor, tedy $(\alpha f + \beta g)'(x) = \alpha f'(x) + \beta g'(x)$ pokud je pravá strana definována \label{poucka:derivace_je_linearni_operator}
		\item  $(f\cdot g)'(x) = f'(x) g(x) + f(x) g'(x)$ pokud je pravá strana definována a $f$ nebo $g$ je spojitá \label{poucka:derivace_soucinu}
		\item  $(f\circ g)'(x) = f'(g(x)) g'(x)$ pokud je pravá strana definována, $g, f$ mají derivaci a $g$ je spojitá \label{poucka:derivace_slozene_fce}
		\item  $\ln(x)' = 1/x$
	\end{enumerate}
	\label{thm:poucky_o_derivacich}
\end{theorem}

