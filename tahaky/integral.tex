\begin{theorem}
	Nechť $F$ je primitivní funkce k $f$
	a nechť $G$ je primitivní funkce k funkci $g$
	na nějakém intervalu $I$.
	Nechť $\alpha, \beta \in \mathbb{R}$ jsou dvě čísla.
	Pak:
	$$\int \alpha f(x) + \beta g(x) \dx = \alpha F(x) + \beta G(x)$$
	\label{thm:integral_linearni_kombinace}
\end{theorem}

\begin{theorem}[O substituci]
	Mějme funkce
	$$\varphi \colon (\alpha, \beta) \rightarrow (a, b)$$
	$$f \colon (a, b) \rightarrow \mathbb{R}$$
	přičemž $\varphi$ má na $(\alpha, \beta)$ vlastní derivaci.

	Nechť $F \colon (a, b) \rightarrow \mathbb{R}$ je primitivní funkcí k $f$ na intervalu $(a, b)$.
	Pak na intervalu $(\alpha, \beta)$ platí, že
	$$\int f(\varphi(t)) \varphi'(t) \dt = F(\varphi(t)) + c$$
	\label{thm:integral_veta_o_substituci}
\end{theorem}

\begin{theorem}[Integrace per partes]
	Nechť jsou funkce $f, g$ spojité na intervalu $(a, b)$ a funkce $F, G$ jsou k nim na $(a, b)$ primitivní.
	Potom i funkce $Fg$, $fG$ mají na intervalu $(a, b)$ primitivní funkci a tamtéž platí identita:
	$$\int f(x) G(x) \dx + \int F(x) g(x) \dx = F(x) G(x) + c$$
	\label{thm:integrace_per_partes}
\end{theorem}

