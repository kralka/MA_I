\begin{definition}[Hromadný bod]
	Reálné číslo $\alpha$ nazveme hromadným bodem posloupnosti $(a_n)$, pokud existuje posloupnost $(b_n)$ vybraná z $(a_n)$, která má limitu $\alpha$.
	Množinu všech hromadných bodů posloupnosti $(a_n)$ značíme $H(a_n)$.
	Dále definujme nejmenší a největší limitu posloupnosti
	$$
			\liminf_{n \rightarrow \infty} a_n = \min(H) \qquad \mbox{a} \qquad \limsup_{n \rightarrow \infty} a_n = \max(H).
	$$ 
	\label{def:hromadny_bod}
\end{definition}

\begin{theorem}[Základní vlastnosti množiny hromadných bodů]
	Množina $H(a_n)$ je neprázdná a je jednobodová právě, tehdy když $(a_n)$ má limitu.
	Hodnoty $\liminf$ a $\limsup$ vždy existují a pokud je posloupnost omezená, tak jsou to vlastní hodnoty.
	(Podívejte se na ekvivalenty těchto vlastností do poznámek z přednášky.)
	\label{thm:vety_o_hromadnych_bodech}
\end{theorem}

