\begin{theorem}[Aritmetika limit funkcí]
	Nechť $a, A, B \in \mathbb{R}^*$,
	nechť $f, g$ jsou funkce definované na nějakém prstencovém okolí $P(a, \Delta)$ bodu $a$,
	nechť platí $\underset{x \rightarrow a}{\lim} f(x) = A$, $\underset{x \rightarrow a}{\lim} g(x) = B$.
	Potom:
	\begin{enumerate}

		\item  $\underset{x \rightarrow a}{\lim} f(x) + g(x) = A + B$, je-li tento součet definovaný
			\label{thm:aritmetika_limit_funkci:soucet}

		\item  $\underset{x \rightarrow a}{\lim} f(x) g(x) = A \cdot B$, je-li tento součin definovaný
			\label{thm:aritmetika_limit_funkci:soucin}

		\item  Nechť je navíc $g$ na nějakém prstencovém okolí bodu $a$ nenulová, pak
			$\underset{x \rightarrow a}{\lim} f(x) / g(x) = A / B$, je-li tento podíl definovaný
			\label{thm:aritmetika_limit_funkci:podil}

	\end{enumerate}
	\label{thm:aritmetika_limit_funkci}
\end{theorem}

