Exponenciálu jste definovali následovně: 
$$e^x = \sum_{n=0}^\infty \frac{x^n}{n!} = 1+x+\frac{x^2}{2}+\frac{x^3}{6}+ \ldots$$
Speciálně jste měli tvrzení bez důkazu, že tato řada konverguje pro každé reálné číslo $x \in \mathbb{R}$.

Přesvědčete se o tom, že
$\lim_{x \rightarrow 0} \frac{e^x -1}{x} =1$ (speciálně budete muset věřit, že operace s nekonečnou řadou jsou ok).

\solution{
	Teď budeme věřit, že operace s nekonečnou řadou jsou ok (zrovna tyhle se dají rozmyslet jednoduše, pokud věříme, že $e^x$ tak jak jsme ho definovali konverguje pro libovolné reálné~$x$).

	Vadí nám, že \uv{dělíme nulou}.
	Vyjdeme z definice exponenciely a upravíme
	$$\frac{e^x -1}{x} = \frac{1+x+\frac{x^2}{2}+\frac{x^3}{6}+ \ldots -1}{x} = \frac{x+\frac{x^2}{2}+\frac{x^3}{6}+ \ldots}{x} =  1+\frac{x}{2}+\frac{x^2}{6}+ \ldots$$
	a nyní dosadíme $x=0$ a dostaneme řadu, která triviálně konverguje k 1.

	O výsledku se můžeme přesvědčit i numericky:
	$$\frac{e^{0,1} - 1}{0,1} = 1,05171$$
	$$\frac{e^{0,01} - 1}{0,01} = 1,00502$$
	$$\frac{e^{0,001} - 1}{0,001} = 1,0005$$
	$$\frac{e^{-0,1} - 1}{-0,1} = 0,951626$$
	$$\frac{e^{-0,01} - 1}{-0,01} = 0,995017$$
	$$\frac{e^{-0,001} - 1}{-0,001} = 0,9995$$
}

