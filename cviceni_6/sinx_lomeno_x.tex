Definujeme $$\sin(x) =  \sum_{n=0}^\infty \frac{(-1)^n x^{2n+1}}{(2n+1)!} = x-\frac{x^3}{3!}+\frac{x^5}{5!}- \ldots$$
a opět věříme tvrzení bez důkazu, že tato řada konverguje pro libovolné $x \in \mathbb{R}$ (a dokonce absolutní hodnota výsledku je nejvýš jedna).

Přesvědčete se, že $\underset{x \rightarrow 0}{\lim} \frac{\sin(x)}{x} = 1$.

\solution{
	Podobně jako v předchozím příkladu
	$$\frac{\sin(x)}{x} = \frac{ x-\frac{x^3}{3!}+\frac{x^5}{5!}- \ldots}{x} =  1-\frac{x^2}{3!}+\frac{x^4}{5!}- \ldots$$
	a dosadíme $x=0$.

	Tento příklad má i intuitivní vysvětlení, když se podíváte na graf $\sin$ hodně blízko nuly, tak je to skoro lineární funkce.
	Poznamenejme, že z definice $\sin$ můžeme i vykoukat, že $\sin(x) \leq x$ pro libovolné $x \geq 0$.
}

