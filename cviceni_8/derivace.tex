Připomeňte si poučky o derivacích (Věta~\ref{thm:poucky_o_derivacich}).
Spočítejte derivace následujících funkcí:
\begin{enumerate}

	\item  $\sqrt{\sin(x)}$

		\solution{
			Jedná se o složenou funkci, vnější je $\sqrt{x} = x^{1/2}$ kterou zderivujeme pomocí \ref{poucka:derivace_monomu}.
			Vnitřní podle poučky.

			$\frac{\cos(x)}{2 \sqrt{\sin(x)}}$
		}

	\item  $\cos^2(x) + \sin^2(x)$

		\solution{
			Derivujeme součet dvou součinů:
			$-2 \sin(x)\cos(x) + 2\sin(x)\cos(x) = 0$.

			Tohle dává smysl, protože si pamatujeme, že $\sin^2(x) + \cos^2(x) = 1$ a derivace konstanty je nula.
		}

	\item  Dle věty o derivaci inverzní funkce ověřte výsledek derivace $\ln(x)$.

		\solution{
			Logaritmus je inverzní funkce exponenciály, tedy platí $\ln\left( e^x \right) = x$.
			Exponenciála je spojitá a ryze monotónní.
			Tedy máme:
			$$(\ln(e^x))' = \frac{1}{e^x}$$
			Pro konkrétní $x = e^y$ dostaneme:
			$$(\ln(x))' = \frac{1}{x}$$
		}

	\item  $\frac{x^2 + 1}{3x}$

		\solution{
			Dle derivace podílu máme:
			\begin{align*}
				\left( \frac{x^2 + 1}{3x} \right)' &= \frac{ (x^2 + 1)'(3x) - (3x)'(x^2 + 1)}{(3x)^2} \\
				&= \frac{ 6x^2 - 3x^2 - 3}{9x^2} \\
				&= \frac{ x^2 - 1}{3x^2}
			\end{align*}
		}

	\item  $x^x$

		\solution{
			Na tuto funkci nemáme tabulku.
			Ale můžeme ji upravit a pak už máme postup, jak derivovat
			$$x^x = (e^{\ln(x)})^{x} = e^{x \ln(x)}$$

			Nyní máme spočítat derivaci složené funkce, vnější funkce je $e^y$, vnitřní je $x \ln(x)$, tedy součin funkcí.
			\begin{align*}
				\left( e^{x \ln(x)} \right)' &= e^{x \ln(x)} \left( x \ln(x) \right)' \\
				&= e^{x \ln(x)} \left( \ln(x) + x (\ln(x))' \right) \\
				&= e^{x \ln(x)} \left( \ln(x) + 1 \right) \\
				&= x^{x} \left( \ln(x) + 1 \right)
			\end{align*}
		}

\end{enumerate}

