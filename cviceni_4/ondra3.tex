Určete hromadné body posloupnosti definované následovně:
$a_1=1$, pro $n \geq 2$ je $a_n = \min \{d \in \mathbb{N}: d \geq 2 \wedge d \backslash n\}$ (poslední znak znamená, že $d$ dělí $n$).

\solution{
	Pro $n \geq 2$ je $a_n$ nejmenší netriviální dělitel čísla $n$, tedy jistě prvočíslo.
	Na druhou stranu, pro prvočíslo $p$ jsou všechny členy $a_{p^k}, k = 1,2,\ldots$ rovny $p$.
	Tedy tvoří konstantní podposloupnost a $p$ je její limitou a tím i hromadným bodem posloupnosti $(a_n)_{n=1}^\infty$. 
}

