Nechť $f, g\colon D \rightarrow \mathbb{R}$, ukažte, že následující funkce jsou spojité:
\begin{enumerate}

	\item  $(f + g)(x) = f(x) + f(x)$

		\solution{
			Dle definice je funkce $f$ spojitá v bodě $a \in \mathbb{R}$, pokud:
			$$\lim_{x \rightarrow a} f(x) = f(a)$$

			Nyní už můžeme jednoduše použít větu o aritmetice limit funkcí~\ref{thm:aritmetika_limit_funkci}: 
			\begin{align*}
				(f+g)(a) &= f(a) + g(a) \\
				&= \lim_{x \rightarrow a} f(x) + \lim_{x \rightarrow a} g(x) \tag{spojitost $f, g$} \\
				&= \lim_{x \rightarrow a} f(x) + g(x) \tag{věta o aritmetice limit funkcí} \\
				&= \lim_{x \rightarrow a} (f+g)(x)
			\end{align*}
		}

	\item  $(f \cdot g)(x) = f(x) g(x)$

		\solution{
			Obdobně, akorát použijeme součinovou část věty o aritmetice limit funkcí~\ref{thm:aritmetika_limit_funkci}.
		}

\end{enumerate}

Nechť $g \colon D_g \rightarrow \mathbb{R}$, $f \colon D_f \rightarrow R_f \subseteq D_g$, pak jejich složení je spojitá funkce:
$$(g \circ f)(x) = g(f(x))$$

\solution{
	Tady budeme chtít použít větu o limitě složené funkce~\ref{thm:limita_slozene_fce}, využijeme toho, že vnější funkce $g$ je spojitá (tedy první podmínka je splněná).

	\vskip3mm
	\hrule
	\vskip3mm
	Předchozí tvrzení je jen implikace, protože složení dvou nespojitých funkcí může být spojité.
	Jako příklad definujme funkci $f \colon \mathbb{R} \rightarrow \mathbb{R}$, která není spojitá, ale složení sama se sebou je spojitá funkce:
	$$f(x) =
	\begin{cases}
		1 & \text{pokud } x \in \mathbb{Q} \\
		0 & \text{pokud } x \in \mathbb{R} \setminus \mathbb{Q}
	\end{cases}
	$$
	Pak ale funkce $(f\circ f)(x) = f(f(x)) = 1$ neboť výsledek první aplikace $f$ je buď jedna nebo nula (obojí je racionální číslo).
	To že $f$ není spojitá si rozmyslete sami.

	Obdobně poznamenejme, že i součet nespojitých funkcí může být spojitá funkce (například předchozí funkce plus její negace).
	Sami vymyslete příklad dvou nespojitých funkcí, jejichž součin je spojitý.
	\vskip3mm
	\hrule
	\vskip3mm
}

S pomocí předchozího dokažte, že následující funkce jsou spojité:

\begin{enumerate}

	\item  $\sin(x) \sqrt{x} - 28 \ln(x)$

		\solution{
			Definiční obor funkce $\sin(x)$ jsou všechna reálná čísla.
			Definiční obor funkce $\sqrt{x}$ jsou všechna nezáporná reálná čísla.
			Definiční obor funkce $\ln(x)$ jsou všechna kladná reálná čísla.
			Definiční obor celé funkce jsou tedy všechna kladná reálná čísla.

			Konstanta $-28$ je spojitá funkce.
			Zbytek je součet dvou součinů spojitých funkcí.
		}

	\item  $e^{\sin(\ln(x))}$

		\solution{
			Exponenciála i $\sin(x)$ jsou definované pro všechna reálná čísla.
			Funkce $\ln(x)$ je definovaná jen pro kladná reálná čísla.
			Definiční obor celé funkce jsou tedy všechna kladná reálná čísla.

			Složení spojitých funkcí je spojitá funkce (a tady máme složení spojité funkce a složení dvou spojitých funkcí).
		}

\end{enumerate}

